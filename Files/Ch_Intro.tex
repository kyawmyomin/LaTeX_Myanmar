\chapter{မိတ်ဆက်}
\label{ch:Introduction}

မြန်မာဘာသာဖြင့် ရေးသား ထားသည် များကို အသုံးပြုရန် ယူနီကုဒ်ကို ထောက်ပံ့သော XeLaTeX ကို အသုံးပြုရန် လိုသည်။ 
%------------------------------------------------------------------------------
\section{TeXstudio ကို Configure လုပ်ခြင်း}

နမူနာ အနေနှင့် အသုံးများသော LaTeX editor တစ်ခုဖြစ်သည့် TeXstudio ကိုသုံးမည်။
အောက်ပါ ဇယား~\ref{tbl_eg} တွင် အသုံးပြုထားသည့် style ဖိုင်များကို ပြထားသည်။

\begin{table}[h]\footnotesize
	\centering	
	\caption{ဖိုင်များ၏ဇယား။}
\begin{tabular}{| c | c | }	
	\hline
	ဖိုင်  & ဖော်ပြချက် \\
	\hline
	burmese.sty & မြန်မာစာ ပြောင်းပေးသည့် ဖိုင်။  \\
	\hline
	lststyles.sty & code listing များ၏ စတိုင်။  \\
	\hline
\end{tabular}
	\label{tbl_eg}
\end{table}

\section{ဖောင့်ပြင်ခြင်း}

အသုံးပြုသည့် မြန်မာဖောင့်ကို ပြင်လိုပါက burmese.sty ဖိုင်တွင် ပြင်ရမည်။
 ဥပမာ Pyidaungsu ဖောင့်သုံးလိုပါက
\begin{lstlisting}[style=myTeX,caption={မူရင်ဖောင့် သတ်မှတ်ချက်။},label={lstMTFont}]
\setmainfont{Myanmar Text}
\end{lstlisting}
 နေရာတွင်
\begin{lstlisting}[style=myTeX,caption={ဖောင့်အမည်ကိုပြင်ခြင်း။},label={lstPDFont}]
\setmainfont{Pyidaungsu}
\end{lstlisting}
 ဟုပြင်နိုင်သည်။ 
